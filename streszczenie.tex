\pdfbookmark[0]{Streszczenie}{streszczenie.1}
%\phantomsection
%\addcontentsline{toc}{chapter}{Streszczenie}
%%% Poni�sze zosta�o niewykorzystane (tj. zrezygnowano z utworzenia nienumerowanego rozdzia�u na abstrakt)
%%%\begingroup
%%%\setlength\beforechapskip{48pt} % z jakiego� powodu by�a male�ka r�nica w po�o�eniu nag��wka rozdzia�u numerowanego i nienumerowanego
%%%\chapter*{\centering Abstrakt}
%%%\endgroup
%%%\label{sec:abstrakt}
%%%Lorem ipsum dolor sit amet eleifend et, congue arcu. Morbi tellus sit amet, massa. Vivamus est id risus. Sed sit amet, libero. Aenean ac ipsum. Mauris vel lectus. 
%%%
%%%Nam id nulla a adipiscing tortor, dictum ut, lobortis urna. Donec non dui. Cras tempus orci ipsum, molestie quis, lacinia varius nunc, rhoncus purus, consectetuer congue risus. 
%\mbox{}\vspace{2cm} % mo�na przesun��, w zale�no�ci od d�ugo�ci streszczenia
\begin{abstract}
Praca przedstawia aplikacj� internetow� skierowan� do lokalnej spo�eczno�ci fitness. Aplikacja ma u�atwia� wsp�prac� pomi�dzy trenerem personalnym, a podopiecznym. Pocz�tek pracy po�wi�cono na projektowanie aplikacji oraz zdefiniowanie jej wymaga�. W kolejnych rozdzia�ach skupiono si� na implementacji, interfejsie u�ytkownika oraz testach.


\end{abstract}
\mykeywords{}

% Dobrze by�oby skopiowa� s�owa kluczowe do metadanych dokumentu pdf (w pliku Dyplom.tex)
% Niestety, zaimplementowane makro nie robi tego z automatu, wi�c pozostaje kopiowanie r�czne.

{
\selectlanguage{english}
\begin{abstract}
Thesis presents web application aimed at the local community interested in fitness training. The application is designed to facilitate cooperation between the personal trainer and the mentee. The beginning of the thesis was devoted to designing the application and defining its requirements. The following chapters focus on implementation, user interface and testing.

\end{abstract}
\mykeywords{}
}
