\chapter{Instrukcja wdro�eniowa}

Dodatek ten przedstawia spos�b uruchomienia aplikacji. Na p�ycie CD w folderze \texttt{dystrubucja} znajduje si� archiwum \texttt{jar} zawieraj�ce backend aplikacji oraz folder o nazwie \texttt{akademia-treningu}, kt�ry zawiera wersj� dystrybucyjn� frontendu. 

Przed uruchomieniem aplikacji, w systemie powinna by� zainstalowana Java w wersji 17 oraz ustawiona zmienna systemowa \texttt{PATH} wskazuj�ca na miejsce instalacji Javy. Opr�cz tego konieczne jest zainstalowanie menad�era pakiet�w npm. Aby uruchomi� backend aplikacji nale�y otworzy� konsol� w miejscu, w kt�rym znajduje si� plik jar i wpisa� komend� \texttt{java -jar akademia-treningu.jar}. Cz�� backendowa aplikacji powinna zosta� uruchomiona. 

W tym samym miejscu w konsoli nale�y wpisa� komend� \texttt{npx http-server akademia-treningu --port=4200}, aby uruchomi� cz�� frontendow� aplikacji. Mo�liwe, �e menad�er pakiet�w poprosi o doinstalowanie pakietu http-server. Nale�y si� zgodzi�. Po poprawnym uruchomieniu mo�na uruchomi� przegl�dark� i przej�� na stron� \url{http://localhost:4200}, na kt�rej znajdowa� si� b�dzie zaprojektowana aplikacja.