\pdfbookmark[0]{Skr�ty}{skroty.1}% 
%%\phantomsection
%%\addcontentsline{toc}{chapter}{Skr�ty}
\chapter*{Skr�ty}
\label{sec:skroty}
\noindent\vspace{-\topsep-\partopsep-\parsep} % Je�li zaczyna si� od otoczenia description, to otoczenie to l�duje lekko ni�ej ni� wyl�dowa�by zwyk�y tekst, dlatego wstawiano przesuni�cie w pionie

\begin{description}[labelwidth=*]
  \item [ERD] (ang.\ \emph{Entity Relationship Diagram})
	\item [JWT] (ang.\ \emph{JSON Web Token})
	\item [FR] (ang.\ \emph{Functional Requirement})
	\item [NFR] (ang.\ \emph{Non-Functional Requirement})
	\item [PYPL] (ang.\ \emph{PopularitY of Programming Language})
	\item [JPA] (ang.\ \emph{Java Persistence API})
	\item [SPA] (ang.\ \emph{Single Page Application})
	\item [POM] (ang.\ \emph{Project Object Model})
	\item [API] (ang.\ \emph{Application Programming Interface})
	\item [DTO] (ang.\ \emph{Data Transfer Object})
	\item [CRUD] (ang.\ \emph{Create, Read, Update, Delete})
\end{description}